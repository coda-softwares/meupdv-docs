\documentclass[12pt, openright, twoside, article, a4paper, english, brazil]{abntex2}

\usepackage{lmodern}
\usepackage[utf8]{inputenc}
\usepackage[T1]{fontenc}
\usepackage{csquotes}
\usepackage{indentfirst}
\usepackage{hyperref}

% ---
% Pacotes glossaries
% ---
\usepackage[subentrycounter,seeautonumberlist,nonumberlist=true]{glossaries}

% ---
% Pacotes de citações
% ---
\usepackage[brazilian,hyperpageref]{backref}	 % Paginas com as citações na bibl
\usepackage[alf]{abntex2cite}

\titulo{MeuPDV}
\autor{Samuel Henrique Oliveira da Silva\and\\
  Jeffersom David da Silva Bomfim\and\\
  Eduardo Silva Lopes}
\data{31 de Maio de 2017}
\instituicao{%
    Instituto Paula Souza
    \par	
    ETEC Heliópolis}
\preambulo{Trabalho de Conclusão de Curso apresentado como pré-requisito para 
    obtenção do Diploma de Técnico de Informática.}

\local{São Paulo}
\tipotrabalho{Trabalho de Conclusão de Curso}
\orientador[Prof.]{Rogério}
\makeatletter
\hypersetup{
pdftitle={\@title},
pdfauthor={\@author},
pdfsubject={\imprimirpreambulo},
pdfkeywords={PALAVRAS}{CHAVE}{EM}{PORTUGUES},
pdfcreator={LaTeX with abnTeX2},
colorlinks=true,
linkcolor=blue,
citecolor=blue,
urlcolor=blue
}
\makeatother

\setlength{\parindent}{1.3cm}
\setlength{\parskip}{0.2cm} % tente também \onelineskip
\setlrmarginsandblock{3cm}{2cm}{*}
\setulmarginsandblock{3cm}{2cm}{*}
\checkandfixthelayout
\bibdata{uni.bib}

\makeindex

\makeglossaries
% --------------------------------------
% - LISTA DE GLOSSÁRIO
% --------------------------------------
\newacronym{pdv}{PDV}{Ponto de Venda}

\begin{document}
\imprimircapa

\imprimirfolhaderosto

\begin{epigrafe}
\vspace*{\fill}
\begin{flushright}
\textit{‘‘Não me envergonho de mudar \\
de ideia por que não me \\
envergonho de pensar".\\
(FREUD. Sigmund)}
\end{flushright}
\end{epigrafe}
\clearpage

\pdfbookmark[0]{\contentsname}{toc}
\tableofcontents*
\cleardoublepage

\textual
\begin{resumo}
\addcontentsline{toc}{chapter}{Resumo}
  Hoje em dia em negócios de varejo de pequeno e médio porte, existe um problema
causado pela falta de conhecimento dos vendedores e proprietários, em relação
aos complexos sistemas, e regulamentações inerentes a administração de um ponto
de venda. Causando uma má admistração por conta da falta de conhecimento e
\ref{gomes1998}
preparo que é geralmente o caso de pequenos proprietários de lojas. Coisas que
presumimos serem garantidas em qualquer loja de varejo, para muitos pequenos
empresários são luxos inviáveis; um dos exemplos mais simples de se dar são,
caixas computadorizados com leitores a laser e softwares e servidores \gls{pdv}.
Qualquer grande varejista possui esse conjunto básico, mas ainda não é uma
realidade ubíqua nem mesmo nos grandes centros metropolitanos do Brasil, quem
dira a partes mais remotas do nosso país sub-desenvolvido. 
  
\vspace{\onelineskip}
\noindent
\textbf{Palavras-chave}: negócio. venda. pequeno porte.
\end{resumo}
\clearpage

\begin{resumo}[Resume]
\begin{otherlanguage*}{english}
Nowdays, in small local bussinesses, there is a problem caused by the lack of 
knowledge of the retail sellers in relation to the complex system that a 
store needs to enter in the real world business. That implies in the lack of a
savings acount that serve as a base, given the nescessities of the 21st century
consumer. For an example, the checkout, it isn't every store that has a fully
automated one, with laser scanners, a computer and a software to manage it. Most
of the time

\vspace{\onelineskip}
\noindent
\textbf{Keywords}: latex. abntex. publication de textes.
\end{otherlanguage*}
\end{resumo}

\clearpage
\postextual
 
 
% ----------------------------------------------------------
% Referências bibliográficas
% ----------------------------------------------------------
\bibliography{uni.bib}

\phantompart
\renewcommand{\glossaryname}{Glossário}
\renewcommand{\glossarypreamble}{Esta é a descrição do glossário. Experimente
visualizar outros estilos de glossários, como o \texttt{altlisthypergroup},
por exemplo.\\
\\}

% ---
% Traduções para o ambiente glossaries
% ---
\providetranslation{Glossary}{Glossário}
\providetranslation{Acronyms}{Siglas}
\providetranslation{Notation (glossaries)}{Notação}
\providetranslation{Description (glossaries)}{Descrição}
\providetranslation{Symbol (glossaries)}{Símbolo}
\providetranslation{Page List (glossaries)}{Lista de Páginas}
\providetranslation{Symbols (glossaries)}{Símbolos}
\providetranslation{Numbers (glossaries)}{Números} 
% ---

% ---
% Estilo de glossário
% ---
% \setglossarystyle{index}
% \setglossarystyle{altlisthypergroup}
 \setglossarystyle{tree}


% ---
% Imprime o glossário
% ---
\cleardoublepage
\phantomsection
\addcontentsline{toc}{chapter}{\glossaryname}
\printglossary
\end{document}

