\documentclass[12pt, openright, twoside, article, a4paper, brazil]{abntex2}

\usepackage{lmodern}
\usepackage[utf8]{inputenc}
\usepackage[T1]{fontenc}
\usepackage{csquotes}
\usepackage[alf]{abntex2cite}
\usepackage{indentfirst}
\usepackage{hyperref}
\bibliography{uni.bib}

\titulo{MeuPDV}
\autor{Samuel H. O. Silva\and Jeffersom D. S. Bomfim\and Eduardo Silva Lopes}
\data{31 de Maio de 2017}
\instituicao{%
    Instituto Paula Souza
    \par	
    ETEC Heliópolis}
\preambulo{Trabalho de Conclusão de Curso apresentado como pré-requisito para 
    obtenção do Diploma de Técnico de Informática.}

\local{São Paulo}
\tipotrabalho{Trabalho de Conclusão de Curso}
\orientador[Prof.]{Rogério}
\hypersetup{
pdftitle={\@title},
pdfauthor={\@author},
pdfsubject={\imprimirpreambulo},
pdfkeywords={PALAVRAS}{CHAVE}{EM}{PORTUGUES},
pdfcreator={LaTeX with abnTeX2},
colorlinks=true,
linkcolor=blue,
citecolor=blue,
urlcolor=blue
}

\setlength{\parindent}{1.3cm}
\setlength{\parskip}{0.2cm} % tente também \onelineskip
\setlrmarginsandblock{3cm}{2cm}{*}
\setulmarginsandblock{3cm}{2cm}{*}
\checkandfixthelayout

\begin{document}
\imprimircapa
\imprimirfolhaderosto

\begin{epigrafe}
\vspace*{\fill}
\begin{flushright}
\textit{‘‘Não me envergonho de mudar \\
de ideia por que não me \\
envergonho de pensar.\\
(FREUD. Sigmund)}
\end{flushright}
\end{epigrafe}
\clearpage

\begin{resumo}
Hoje em dia, em mercados da comunidade, há um problema decorrente da falta de 
conhecimento dos vendedores em relação ao complexo sistema que um mercado 
necessita para poder embarcar no mundo real de hoje. Este implica na falta de 
uma poupança que sirva como base diante as necessidades do consumidor de hoje 
em dia. Por exemplo, o caixa. Não são todas as lojas que têm um todo 
automatizado, com scanner a laser, um computador para administrar seu banco e 
uma aplicação que gerencie seu negócio. Nem tanto, um grande número de 
empregados para cuidar de seu trabalho árduo deixando-o focar no mais 
importante. Diante desse contexto podemos notar como é delicado cuidar desta 
área devido a suas complicações que todos acabam encarando quando se é 
iniciante. Dentre elas está uma certificação, de ter os documentos oficiais em 
dia.

\vspace{\onelineskip}
\noindent
\textbf{Palavras-chave}: neócio. venda. pequeno porte.
\end{resumo}
\clearpage

\begin{resumo}[Resume]
\begin{otherlanguage*}{english}
Nowdays, in small local bussinesses, there is a problem caused by the lack of 
knowledge of the retail sellers in relation to the complex system that a 
store needs to enter in the real world business. That implies in the lack of a
savings acount that serve as a base, given the nescessities of the 21st century
consumer. For an example, the caixa. Não são todas as lojas que têm um todo 
automatizado, com scanner a laser, um computador para administrar seu banco e 
uma aplicação que gerencie seu negócio. Nem tanto, um grande número de 
empregados para cuidar de seu trabalho árduo deixando-o focar no mais 
importante. Diante desse contexto podemos notar como é delicado cuidar desta 
área devido a suas complicações que todos acabam encarando quando se é 
iniciante. Dentre elas está uma certificação, de ter os documentos oficiais em 
dia.

\vspace{\onelineskip}
\noindent
\textbf{Keywords}: latex. abntex. publication de textes.
\end{otherlanguage*}
\end{resumo}

\end{document}

