\documentclass[12pt, openright, twoside, article, a4paper, english, brazil]{abntex2}

\usepackage[utf8]{inputenc}
\usepackage{lmodern}
\usepackage[T1]{fontenc}
\usepackage[backend=biber]{biblatex}
\usepackage{csquotes}
\usepackage{indentfirst}
\usepackage{hyperref}
\usepackage[toc]{glossaries}
\usepackage{blindtext}
\addbibresource{uni.bib}

\titulo{MeuPDV}
\autor{Samuel Henrique Oliveira da Silva\and\\
  Jeffersom David da Silva Bomfim\and\\
  Eduardo Silva Lopes}
\data{31 de Maio de 2017}
\instituicao{%
    Instituto Paula Souza
    \par       
    ETEC Heliópolis}
\preambulo{Trabalho de Conclusão de Curso apresentado como pré-requisito para 
    obtenção do Diploma de Técnico de Informática.}

\local{São Paulo}
\tipotrabalho{Trabalho de Conclusão de Curso}
\orientador[Prof.]{Rogério}
\makeatletter
\hypersetup{
pdftitle={\@title},
pdfauthor={\@author},
pdfsubject={\imprimirpreambulo},
pdfkeywords={PALAVRAS}{CHAVE}{EM}{PORTUGUES},
pdfcreator={MeuPDV},
colorlinks=true,
linkcolor=black,
citecolor=blue,
urlcolor=blue
}

\setlength{\parindent}{1.3cm}
\setlength{\parskip}{0.2cm} % tente também \onelineskip
\setlrmarginsandblock{3cm}{2cm}{*}
\setulmarginsandblock{3cm}{2cm}{*}
\checkandfixthelayout

% -----------------------------------------------
% - Glossário
% ----------------------------------------------
\makeglossaries
\newacronym{pdv}{PDV}{Ponto de venda}

\begin{document}
\imprimircapa{}
\imprimirfolhaderosto{}

\chapter*{Resumo\label{resumo}}
Esse texto vem apresentar o software MeuPDV criado para o trabalho de conclusão
de curso da ETEC heliópolis 2017, o maior objetivo do projeto é prover uma
ferramenta descomplicada e eficiente para a adminstação de pequenas lojas. Tendo
como público alvo pequenos empresários e empreendedores do setor de vendas em
geral, um dos maiores pontos fortes da nossa solução é o quase inexistente
investimento inicial, tanto no sentido de equipamento, já que ferramentas como
lasers, mesas de caixa, computadoes, não são nescessárias e também pela
popularidade massiva do sistema operacional android que permite utilizar os
dispositivos que os funcionários teoricamente já possuem.

\vspace{\onelineskip}
\noindent
\textbf{Palavras-chave}: negócio. venda. pequeno porte.
\clearpage

\tableofcontents
\cleardoublepage
 
\textual
\chapter{Introdução}
Hoje em dia em negócios de varejo de pequeno e médio porte, existe um problema
causado pela falta de conhecimento dos vendedores e proprietários, em relação
aos complexos sistemas, e regulamentações inerentes a administração de um ponto
de venda. Causando uma má admistração por conta da falta de conhecimento e
preparo que é geralmente o caso de pequenos proprietários de
lojas. Coisas que presumimos serem garantidas em qualquer loja de varejo, para
muitos pequenos empresários são luxos inviáveis; um dos exemplos mais simples de
se dar são, caixas computadorizados com leitores a laser e softwares e
servidores \gls{pdv}. Qualquer grande varejista possui esse conjunto básico, mas
ainda não é uma realidade ubíqua nem mesmo nos grandes centros metropolitanos do
Brasi, quem  dira a partes mais remotas do nosso país sub-desenvolvido.  

\postextual
\clearpage
\printbibliography

\end{document}

